\documentclass[conference]{IEEEtran}
\usepackage{float}
\usepackage{graphicx}
\usepackage[final]{hyperref} % adds hyper links inside the generated pdf file
\usepackage{wrapfig}
\usepackage{booktabs}
\hypersetup{
	colorlinks=true,       % false: boxed links; true: colored links
	linkcolor=blue,        % color of internal links
	citecolor=blue,        % color of links to bibliography
	filecolor=magenta,     % color of file links
	urlcolor=blue
}

%package list
\usepackage[
    backend=bibtex,
    style=numeric,
    sorting=none,
    maxcitenames=2
]{biblatex}


%\addbibresource{main.bib}
\bibliography{references.bib}


\begin{document}

\title{Using {\it imo.im} in Forensic Investigations \\\vspace{5mm} \large  \today}
\author{
\IEEEauthorblockN{Robin Klusman}
\IEEEauthorblockA{
Security and Network Engineering \\
University of Amsterdam \\
robin.klusman@os3.nl}
\and
\IEEEauthorblockN{Luc Gommans}
\IEEEauthorblockA{
Security and Network Engineering \\
University of Amsterdam \\
os3-ccf@lucgommans.nl}
}
\maketitle
\thispagestyle{plain}
\pagestyle{plain}

\begin{abstract}
    This paper presents a forensic investigation of the imo.im Android
    application. Imo is a messaging and voice and video calling application with
    over half a billion installations worldwide. %Todo
\end{abstract}

\section{Introduction}

In recent years, mobile communication tools have gained massively in popularity.
Many mobile device users nowadays use internet based third party messaging and
calling applications over the small message service (SMS) and regular phone
calls provided by telecom companies.  The advantage of these internet based
alternatives is often the more extensive features an lower cost compared to SMS.
It is therefore unsurprising that many such applications have over hundreds of
millions of downloads, making the use of them widespread. This fact is important
for forensics purposes because there are reasonable odds that a suspect has
installed and possibly used at least one such application to send or receive
sensitive data relevant to the case. The investigation into what data useful in
a forensic investigation can be gathered from such applications, therefore
becomes very relevant.

One example of such an application is {\it imo.im}, which currently has over
half a billion installs on the Google Play Store. Fewer than ten messaging
applications have over half a billion installs as of October
2017\cite{wiki-gplay-popular}. It appears most popular in southern Asia and the
middle east, though there are users worldwide\footnotemark. This application
requests permission to track the user's location, making it even more
interesting for forensic purposes. If the application indeed tracks and stores
a log of a user's location data, such data could be valuable in court to for
instance validate or invalidate a suspect's alibi.

The remainder of this paper is structured as follows: the next section briefly
introduces the features of imo. Section \ref{sec:researchq} describes our main
research question and the defined sub-questions. In Section \ref{sec:ethics} we
discuss the ethical implications of our work. Section \ref{sec:relwork} gives an
overview of the prior work done in this area of research. In Section
\ref{sec:method} we discuss our methods and experiment setup. Section
\ref{sec:network} describes our findings from a network analysis while Section
\ref{sec:storage} describes our analysis of the mobile device's local storage.
%TODO

% Or do we want to cite this instead?
\footnotetext{Via Google Trends: \url{https://trends.google.com/trends/explore?q=imo}}
%TODO this is heavily influenced by the acronym imo, meaning in my opinion.


\section{Overview of Imo}

As stated previously, imo is a messaging and voice and video calling application
developed for Android, Mac OS X, iOS and Windows \cite{imo}. This description
focusses on the Android version of imo, as the version for other platforms were
not used for this paper.

Imo requires users to log in using their phone number, and each login is
verified through an SMS code sent to that number. There is no difference between
logging into an existing account (i.e. phone number known used previously for
imo) and creating a new account (i.e. phone number not previously used for imo),
as imo uses the same procedure for both. There are no other types of logins
possible, e.g. there is no option to use a username and password to log in.

Contacts in imo are added through your Android phone book. You can send those
contacts text messages, or place (video) calls. You can also share a `story',
which is a graphical message shared with all your contacts. It can consist of a
picture, short video, drawing, etc. Contacts can view your stories and send a
response. These responses appear in the chat history as a reply, referencing
your story.

When chatting, the message you are typing is immediately visible to the other
party, before even sending it. This feature is called `Real time chat'. Sending
a message makes it final: it cannot be edited afterwards, and one can only
delete it locally.

Imo also supports group chats and group calls. Compared to regular chats, we
did not find these to have any special features.

Finally, imo supports what are called `stickers': large, animated emoticons.
These are placed in a chat as a message by themselves.


\section{Research Questions}\label{sec:researchq}

Our main research question is:
{\it What information can be gathered from the `imo' messenger app for Android?}

This main research question divides in four sub-questions:

\begin{enumerate}
    \item What data is identifiably sent over the network?
    \item What message data is available locally?
    \item What location data is available locally?
    \item Do any cached files or images exist locally?
\end{enumerate}


\section{Ethical Considerations}\label{sec:ethics}

For this paper we will use a test environment to examine the data that can be
extracted from the imo android app. We will not be using any real user data and
use only designated test devices to make sure we do not invade or compromise the
privacy of any real imo users.


\section{Related Work}\label{sec:relwork}

No prior work has been done on the imo messaging app for any platform, the only
mention of imo in academic literature is in the paper by \citeauthor{zhu}, which
does a more general review of several group messaging applications and does not
conduct an analysis of what data can be acquired for use in a criminal
investigation \cite{zhu}.

Fortunately academic research does exist on the forensic investigation of other
messaging applications. \citeauthor{mahajan2013forensic} conduct a forensic
analysis of the Android messaging applications Viber and Whatsapp
\cite{mahajan2013forensic}. The aim of their paper is to find whether the data
of those applications, such as sent and received messages and images, are stored
locally in a way that they can be extracted from the device.
\citeauthor{mahajan2013forensic} managed to find and extract several artifacts
relevant to a forensic investigation, such as sent and received messages and
contact phone numbers.

\citeauthor{walnycky2015network} conduct an extensive investigation of 20
popular Android messaging applications \cite{walnycky2015network}. In their
paper they investigate what data is can be retrieved by investigating the mobile
device's local storage, the applications' server storage and network traffic
analysis in a forensic investigation for each application. Network traffic is
analysed using three different tools, Wireshark, NetworkMiner and NetWitness
Investigator, for the purpose of cross referencing the results. Server stored
data was investigated by recovering links to the applications' servers from the
network traffic analysis. Locally stored data was acquired and investigated
using Micro-systemation's {\it XRY} and verified by also using {\it Helium
backup} in combination with {\it Android backup extractor} and an sqlite
database browser.  In their paper, \citeauthor{walnycky2015network} state that 4
of the applications leaked no data when subjected to their investigative
methods.  However, their study did find that 16 applications do leak data either
through network communication or local/server storage.

A forensic investigation of WhatsApp is also conducted by
\citeauthor{anglano2014forensic} \cite{anglano2014forensic}, who uses a virtual
Android environment created with {\it YouWave} to investigate what data can be
acquired from WhatsApp through a device's local storage. Their study shows that
it is very feasible to acquire WhatsApp data from local storage, since data such
as sent and received messages and contact information is stored unencrypted on
the local storage.

\subsection{Experimental Setup}\label{sec:setup}

For our analysis of imo we used two android phones, a Sony Xperia Z (C6603) and
an Alcatel Pixi 4, both of which run Android version 6.0.1. The imo application
has versionCode 0x709 and the versionName is `9.8.000000009731', according to
the \texttt{AndroidManifest.xml} file inside the apk. The SHA-256 hash value of
the apk file is \texttt{61972382 e73a4614 0cae38a0 3eae8b63 9f6e42aa 9d47368d
f207a08b a6f097ae}.


\section{Methods}\label{sec:method}

To investigate what forensically interesting data can be obtained through imo,
we first focus on analysing network traffic. For this analysis, we run a tcpdump
locally on the phone, writing the data in pcap format to a file. We then download
the data from the phone and inspect it using wireshark afterwards. The
results of this are discussed in Section \ref{sec:network}.

The investigation of the local storage is done by first imaging the entire local
storage of the device and then mounting this image read only on our linux
system. Analysis of the image was done manually, the results of this
investigation are discussed in Section \ref{sec:storage}.


\subsection{Experimental Setup}

For our analysis of imo we used two android phones, a Sony Xperia Z (C6603) and
an Alcatel Pixi 4, both of which run Android version 6.0.1. We used imo
version 9.8.000000009671 for our testing.

\section{Network Communication Analysis}\label{sec:network}

Upon analysing the tcpdump in Wireshark, we found that the regular network
communication (e.g. chats) use TCP ports 443, 5223 and 5228. We have not been
able to identify why the app sometimes uses one port over the other. In the APK,
we find references to those three ports together, so we are confident that there
are no other TCP ports in use.  Searching online for those ports, we find that
these ports overlap with WhatsApp, C2DM (now-defunct server-push platform
provided by Google, using port 5228) and Apple's push implementation (APNS,
using port 5223). Most likely, imo uses those ports because they are commonly
used by mobile devices for core operating system features (server push) or other
popular applications. Network administrators are unlikely to block those ports.

The downloading of stickers is done over plain HTTP. Anyone observing the
network traffic can therefore see which emoticons are used by a device. It is
not visible, however, which chat a sticker was used in.
%TODO: ^downloaded only once? What about caching?

UDP is used for voice and video calls, using various ports. The choice of UDP
port seems random for the server, though we have also twice observed the server
to use UDP port 443 which seems too specific to be a coincidence, particularly
because it happened twice. The lowest port observed, other than 443, is 29 879.
Possibly a port is picked above a certain number to avoid conflicts with
existing services.

The Android client also appears to choose ports at random, though the lowest we
have observed was 33 419. We theorize that a random port is chosen above 32 000
to avoid using another application's port.

The clients involved in a call will try to reach each other directly for voice
and video calls. The communication goes through imo's servers by default, but
the app keeps trying to communicate peer to peer throughout the whole session
(it sends a packet twice per second). If this succeeds, it will switch to peer
to peer communication. Calls continue to work if the internet connection is
disrupted, but the users are on the same LAN.


\subsection{Communication Format}

Any stickers used in chats are downloaded over TCP port 80, using HTTP/1.1. A
sticker is a zip archive with one image, in png format, per frame of the
animation. The filenames are sequential numbers: \texttt{1.png}, \texttt{2.png},
etc. When the user is picking a sticker, the previews are downloaded as a single
image over HTTP (rather than the full animation in a zip archive).

When the app chooses to use port 443 for any part of its communication, the
connection is secured using TLSv1.2. The client offers 20 cipher suites, among
which obsolete ciphers such as \texttt{TLS\_RSA\_WITH\_RC4\_128\_SHA} and
\texttt{TLS\_RSA\_WITH\_AES\_128\_CBC\_SHA}, though this is probably dependent
on the operating system. In our case, the server picked
\texttt{TLS\_RSA\_WITH\_AES\_128\_GCM\_SHA256}, which is labeled by Mozilla as a
cipher suite of intermediate strength\cite{moz-tls}.

When the app chooses to use one of the two alternative TCP ports (5223 or 5228),
the connection is encoded using an unknown method. The data is binary and could
be compressed, encrypted or both. When sending multiple kilobytes of repeating
characters, the observed traffic is much smaller, leading us to believe the
data is compressed. It is unknown whether the traffic is also encrypted.

We have not been able to identify the protocol used for UDP traffic. For the
traffic which is sent over port 443, we tried to decode it using Wireshark as
DTLS and QUIC traffic, but neither was successful. The traffic also does not
contain any strings, such as the common name or issuer name from a certificate,
which should have been visible for normal TLS or QUIC traffic.


\section{Local Storage Analysis}\label{sec:storage}

After mounting the mobile device disk image, we found that most of the data used
by imo is stored in the application's own data directory:
\texttt{/sdcard/Android/data/com.imo.android.imoim}. Imo also stores some data
in \texttt{/sdcard/DCIM/imo} and \texttt{/sdcard/IMO} however, only video and
picture data used in conversations or stories are stored here.

In the data directory, we can find all our messages and contacts in the sqlite3
file \texttt{databases/imofriends.db}. For all contacts, both their name and a
numeric \texttt{buid} is stored. Group chats are also stored with their
name/identifier but use a non-numeric \texttt{buid}.

The table `messages' contains all messages, including whether a message was
delivered to the recipient's device and whether a message was read (the columns
\texttt{message\_state} and \texttt{message\_read} are set to `1' when
delivered and read, respectively).

The table `call\_timestamps' contains a convenient log of all calls placed
through the imo application. Only the start time is noted, together with the
contact which was called, not the end time or duration. This timestamp is in
unix format, in nanoseconds, in the UTC timezone. Missed calls are placed in
the `messages' table, as a chat message to the respective user.

The table `friends' contains your contacts. These are all the people you can
message and call. In contrast, the tables `imo\_phonebook', `phone\_numbers',
and `phonebook\_entries' contain the contacts of whom you have a number; these
contacts are the ones also in your Android contact list. It appears that you
are only able to see stories of the users of whom you have the number.

Outside of this database, other files also contain usage information. The JSON
file \texttt{files/brefs.json} contains a field named
\texttt{LAST\_APP\_OPEN\_TS}, which contains a timestamp of when the app was
last opened.


\subsection{Logged-in User Data}

When signing up or logging in, the phone number entered is stored in
\texttt{files/VerificationPrefs.json}, even before the confirmation code sent
through SMS is entered to confirm the number.

After confirming the number, it is also stored in \texttt{files/bprefs.json}.
The JSON file contains a field named \texttt{GET\_MY\_PROFILE}, which is another
JSON string containing a field named \texttt{phone}.

This file also contains the fields \texttt{SIGNUP\_DATE} and
\texttt{SIGNUP\_TIME}, which contain the first time someone logged in with this
phone number and the last time you logged in on this device with this number,
respectively.


\subsection{Removed Data}

When you delete a message from a chat in imo, the application deletes it from
the local database's `messages' table. Similarly, when you delete one of your
stories, it is truly deleted from the `stories' table. However, in the binary
data, we can still find the original message. By parsing the sqlite format, one
might be able to find other fields belonging to the message such as the
corresponding timestamp.

When removing a story, references to the story are not deleted. For example
when one replies to a story, an entry is created in the `messages' table which
refers to it by its object id. These rows are not deleted upon deletion of the
original story. When attempting to view the story in the application, it shows
the story as deleted.

When removing your account on imo, most of the local database tables are
cleared. The `call\_timestamps' table is, however, not cleared and still
contains all previous calls made on this device. Other tables, such as
`messages' and `friends', are cleared. Like with deleted messages, however, the
original data can still be found in the binary data of the sqlite file.

\section{Results}\label{sec:results}

\section{Discussion}\label{sec:disc}

\section{Conclusions}\label{sec:conc}

\section{Future Work}\label{sec:futwork}


% RANDOM NOTES
% ============
%
% - One can log in multiple times with the same account from multiple devices, concurrently
% - TODO: any sort of cache of Real-time-chat? People might chat through that
%   and thereby never log any messages.


% \section{Methods and Scope}
%
% We will do both static and dynamic analysis of the application. We limit
% ourselves to the investigation of the imo Android application and do not look
% into the iOS, MacOS or Windows versions. We will look at the stable release,
% not any (potentially volatile) beta releases.
%
% The main point of interest is network traffic and stored data. If network data
% is not (correctly) encrypted or otherwise revealing, this could be useful in an
% active investigation. Stored data is useful for analysis after confiscating a
% device.
%
% To access both the image and network traffic, we will run the application in an
% Android emulator. This allows the host to capture traffic, as well as mount the
% disk image.


\section{Discussion}

% r/therewasanattempt -- I'm not quite sure if this is a good way to write the discussion section, but I tried something at least :-)

Imo could encrypt the local storage, or it could keep private data only on the
server and delete it from the local device. Instead, it makes no attempt at
either: all data is locally accessible without any obfuscation or encryption.

The databases are also not properly cleared upon deletion of the account: most
tables are emptied, but the database is not recreated. This creates two
problems: the application developers might forget to empty one of the tables
(it appears the call history table has been forgotten) and it causes all data
to be retained in the binary sqlite data file until overwritten. We would
expect an application with private data (such as private chats) to completely
re-create the database, or a truly privacy-conscious to securely wipe the
database file before re-creation.

This is not wholly unexpected: imo does not claim to be privacy-focused, and
other apps should not be able to access the data folder, so it should be
reasonably secure. Still, simple measures could have been taken to make it more
secure.

The accessibility of the data also makes it easy for users to alter their own
data. One could make it appear as if someone else was logged into the device,
or manually insert messages into the chat history.

%TODO: finish the section


\section{Conclusion}

%TODO this


\printbibliography

\end{document}

