\documentclass[conference]{IEEEtran}
\usepackage{float}
\usepackage{graphicx}
\usepackage[final]{hyperref} % adds hyper links inside the generated pdf file
\usepackage{wrapfig}
\usepackage{booktabs}
\hypersetup{
	colorlinks=true,       % false: boxed links; true: colored links
	linkcolor=blue,        % color of internal links
	citecolor=blue,        % color of links to bibliography
	filecolor=magenta,     % color of file links
	urlcolor=blue
}

%package list
\usepackage[
    backend=bibtex,
    style=numeric,
    sorting=none,
    maxcitenames=2
]{biblatex}


%\addbibresource{main.bib}
\bibliography{references.bib}


\begin{document}

\title{Using {\it imo.im} in Forensic Investigations \\\vspace{5mm} \large  \today}
\author{
\IEEEauthorblockN{Robin Klusman}
\IEEEauthorblockA{
Security and Network Engineering \\
University of Amsterdam \\
robin.klusman@os3.nl}
\and
\IEEEauthorblockN{Luc Gommans}
\IEEEauthorblockA{
Security and Network Engineering \\
University of Amsterdam \\
os3-ccf@lucgommans.nl}
}
\maketitle
\thispagestyle{plain}
\pagestyle{plain}


\section{Introduction}

In recent years, mobile communication tools have gained massively in popularity.
Many people nowadays use internet based third party messaging applications over
the small message service (SMS) provided by telecom companies.  Many such
applications have over hundreds of millions of downloads, making them reasonably
widespread. This fact is important for forensics purposes because there are
reasonable odds that a suspect has installed and possibly used at least one such
application. The investigation into what data useful in a forensic investigation
can be gathered from such applications, therefore becomes very relevant.

One example of such an application is {\it imo.im}, which currently  has over
half a billion installs. Additionally, this application requests permission to
track the user's location, making it even more interesting for forensic
purposes. If the application indeed tracks and stores a log of a user's location
data, such data could be valuable in court to for instance validate or
invalidate a suspect's alibi.

The remainder of this paper is structured as follows: Section \ref{sec:II}
describes our main research question and the defined sub-questions. In Section
\ref{sec:III} we discuss the ethical implications of our work. Section
\ref{sec:IV} gives an overview of the prior work done in this area of research.
%TODO

\section{Research Questions}\label{sec:II}

Our main research question is:
{\it What information can be gathered from the `imo' messenger app for Android?}

This main research question divides in four sub-questions:

\begin{enumerate}
    \item What data is identifiably sent over the network?
    \item What message data is available locally?
    \item What location data is available locally?
    \item Do any cached files or images exist locally?
\end{enumerate}


\section{Ethical Considerations}\label{sec:III}

For this paper we will use a test environment to examine the data that can be
extracted from the imo android app. We will not be using any real user data and
use only designated test devices to make sure we do not invade or compromise the
privacy of any real imo users.


\section{Related work}\label{sec:IV}

No prior work has been done on the imo messaging app for any platform, the only
mention of imo in academic literature is in the paper by \citeauthor{zhu}, which
does a more general review of several group messaging applications \cite{zhu}.

Fortunately academic research does exist on the forensic investigation of other
messaging applications. \citeauthor{mahajan2013forensic} conduct a forensic
analysis of the Android messaging applications Viber and Whatsapp
\cite{mahajan2013forensic}. %TODO


% \section{Methods and Scope}
%
% We will do both static and dynamic analysis of the application. We limit
% ourselves to the investigation of the imo Android application and do not look
% into the iOS, MacOS or Windows versions. We will look at the stable release,
% not any (potentially volatile) beta releases.
%
% The main point of interest is network traffic and stored data. If network data
% is not (correctly) encrypted or otherwise revealing, this could be useful in an
% active investigation. Stored data is useful for analysis after confiscating a
% device.
%
% To access both the image and network traffic, we will run the application in an
% Android emulator. This allows the host to capture traffic, as well as mount the
% disk image.





\printbibliography

\end{document}

