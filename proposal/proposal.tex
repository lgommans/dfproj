\documentclass{article}

\usepackage[]{float}
\usepackage[utf8]{inputenc}
\usepackage[]{hyperref}
\usepackage[hyperref=true,
    natbib=true,
    style=numeric,
    backend=bibtex]{biblatex}
\bibliography{references.bib}

\setlength{\parindent}{0pt}
\setlength{\parskip}{\baselineskip}

\title{Using {\it imo.im} in forensic investigations \\
	\vspace{0.3cm}
	{\large CCF Project Proposal}
}
\date{November 16, 2017}
\author{Robin Klusman, Luc Gommans}

\begin{document}
\maketitle

\section{Introduction}

In recent years, mobile communication tools have gained massively in
popularity. Some of these applications have more hundreds of millions of
downloads. This is important for forensics purposes because there are
reasonable odds that a suspect has installed and possibly used one such
application. For example, {\it imo.im} has over half a billion installs and
requests location permissions: if this application logs one's location, this
could be helpful for an investigation to either confirm alibies or to call
false ones into question.


\section{Research Questions}

Our main research question is:
{\it What information can be gathered from the `imo' messenger app for Android?}

This main research question divides in five sub-questions:

\begin{enumerate}
	\item Message data
	\item Location data
	\item Cached files/images
\end{enumerate}


\section{Methods and Scope}

We will do both static and dynamic analysis of the application. We limit
ourselves to the investigation of the imo Android application and do not look
into the iOS, MacOS or Windows versions. We will look at the stable release,
not any (potentially volatile) beta releases.

The main point of interest is network traffic and stored data. If network data
is not (correctly) encrypted or otherwise revealing, this could be useful in an
active investigation. Stored data is useful for analysis after confiscating a
device.

To access both the image and network traffic, we will run the application in an
Android emulator. This allows the host to capture traffic, as well as mount the
disk image.


\section{Planning}

\begin{tabular}[H]{ | l | p{10.2cm} | }
	\hline
	\textbf{Week} & \textbf{Activity} \\
	\hline 1 & todo \\
	\hline 2 & todo \\
	\hline 3 & todo \\
	\hline 4 & todo \\
	\hline 5 & Finailze report and prepare presentation \\
	\hline
\end{tabular}


\section{Ethical Considerations}

For this project we will use a test environment to examine the data that can be
extracted from the imo Android app, we will not be using any real user data and
therefore do not invade the privacy of any real imo users. 


\section{Related work}

There appears to be no previous, academic work into the security or forensics
implications of imo. The only consideration in primary literature appears to be
an article from 2012 which generally reviews group messaging
services\cite{zhu}.


\printbibliography

\end{document}

